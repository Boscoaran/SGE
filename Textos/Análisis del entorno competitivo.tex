Usando las cinco fuerzas de Porter podemos conocer el nivel de competencia que presenta un sector y gracias a ello determinar la estrategia que mejor se adapte a la situación. 

\begin{enumerate}
    \item \textbf{Amenaza de entrada de nuevos competidores potenciales:}
    
    Este apartado analiza la situación del mercado y las barrearas de entrada para nuevos competidores.

    En el montaje de ordenadores el producto principal por el que se paga son los componentes del ordenador más que el montaje de ellos. Debido a esto, este tipo de empresas tiene una gran parte que puede considerarse de distribución. Esto conlleva a que para poder ofrecer el mejor producto al mejor precio es necesario recibir las mejores ofertas de los proveedores. De esta característica surge la barrera de obtener componentes y dispositivos a buen precio y de gran variedad, es decir, conseguir los artículos al mejor precio posible y tener acceso a los últimos modelos.

    \item \textbf{Amenaza de la aparición de productos sustitutivos:}
    
    En este apartado se analiza el riesgo de que un nuevo producto se haga con el mercado y sustituya el producto de nuestra empresa.

    Por el momento no parece posible la aparición de un producto sustitutivos del ordenador de sobremesa. Si bien los portátiles y tablets pueden llevarse una porción del mercado, ambos carecen de la capacidad y posibilidades que ofrece un ordenador de sobremesa y además ambos están disponibles como productos de nuestra empresa.

    \item \textbf{Poder de negociación de los proveedores:}
    
    Este apartado estudia la relación entre los proveedores y nuestra empresa, centrándose en el poder que tienen los proveedores a la hora de poner los precios de los productos.

    En España existen muchos proveedores de componentes para ordenadores y artículos del sector de la informática. Esto es una ventaja para nosotros ya que evita que los proveedores pidan precios excesivamente altos por sus productos al tener que repartirse el mercado y ganar clientes. Como se ha mencionado en el primer apartado conseguir los componentes con los que vamos a trabajar a un buen precio es una aspecto importante, nos interesa por ello tener una buena relación con nuestros proveedores y evitar que lideren las negociaciones de nuestras compras.

    \item \textbf{Poder de negociación de los clientes:}
    
    Similar al anterior apartado, este analiza el poder de negociación de los clientes sobre nuestro sector.

    En nuestro caso, al igual que sucede con los proveedores, los clientes tienen un gran poder de negociación ya que España cuenta con muchas empresas que ofrecen nuestros productos y servicios. Debemos de tener en cuenta que el precio de nuestro producto esta sujeto a la disponibilidad de componentes. Como en los últimos meses se ha observado, la escasez de microchips y otros componentes como tarjetas gráficas ha provocado subidas en los precios. Esto enfatiza la importancia de tener buenos proveedores en este sector.

    \item \textbf{Rivalidad entre los competidores del sector:}
    
    El último apartado analiza las empresas rivales del sector.

    En España el sector esta principalmente compuesto por: PCComponentes, APP Informática, Aussar, Wipoid, Alternate y PCBox. Dependiendo de lo que el cliente busque cada una esta orientada a algo en especial (gaming, trabajo, componentes poco comunes, gamas...). Principalmente podemos dividir el mercado en dos categorías: online y física. Nuestra empresa se centrará en el mercado online ya que consideramos que nuestro producto principal no necesita de la presencia física del cliente sino de un buen servicio online que compare los componentes y ofrezca las mejores opciones.

\end{enumerate}