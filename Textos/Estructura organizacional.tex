Existen diferentes tipos de estructuras organizacionales, entre las que elegiremos la estructura funcional. Esta forma cuenta con diferentes ventajas y desventajas que enumeraremos a continuación:

\begin{itemize}
    \item Cada empleado se encarga exclusivamente de sus tareas, aumentando la eficiencia y la productividad de cada trabajador.
    La supervisión de un área es más sencilla.
    \item La comunicación es mucho más directa al no haber intermediarios de por medio, ayudando a la eficiencia del equipo.
    \item Se reduce la presión sobre el individuo, ya que es más sencillo delegar y compartir responsabilidades.
    \item Pueden generar conflictos de autoridad y aumentar la rivalidad entre compañeros, ya que pueden intentar establecer su enfoque.
    \item Puede existir multiplicidad de objetivos, ocasionándose conflictos en las funciones generales de la organización.    
\end{itemize}

Nuestra empresa contará con un esquema donde el CEO (Chief Executive Officer) de la empresa es el cabeza de la junta directiva apoyado por los demás integrantes de la misma. De este departamento directivo colgarán los demás, que, a continuación, son detallados:

\textbf{Departamento de márketing:} Es uno de los pilares más importantes de la empresa, debido a la necesaria promoción de la organización para darse a conocer, por ello, este departamento se integra en el resto de departamentos. El encargado de márketing será el principal responsable y director de este departamento, cuya función será diseñar e implementar un plan de marketing que se llevará a cabo en un ambiente innovador y dinámico.

\textbf{Departamento de Recursos Humanos:} Este departamento gestiona todo lo relacionado con las personas que trabajan en la empresa. Podemos encontrar el área de contratación y formación del personal, dos de los objetivos más importantes del departamento. Nuestro departamento de Recursos Humanos estará formado por nuestro CHRO(Chief Human Resources Officer) cuya función es impulsar políticas de recursos humanos alineadas con los valores y estrategia de la empresa, y también contaremos con dos equipos, uno encargado de la selección y formación del personal y otro encargado de la gestión del personal.

\textbf{Departamento de atención al cliente:} Establece conexión entre la empresa y todos nuestros clientes post o pre compra, estableciendo una comunicación fluida mediante vía telefónica, por correo electrónico, mensajería o soporte técnico. Contaremos con un ATC manager, que será el encargado de asegurar un funcionamiento eficaz del departamento. Junto a él, nuestro departamento lo formará un grupo de agentes entre los cuales destacaremos unos líderes encargados de atender todas las necesidades que impidan que se desarrolle la función del departamento de forma correcta.

\textbf{Departamento de Logística y Almacén:} otro de los pilares más importantes de la empresa. Este departamento se encarga de guardar, proteger, conservar y enviar cada uno de nuestros productos que están en catálogo para su venta. Entre los objetivos del departamento destaca el mantenimiento de un servicio de gran calidad, garantizando la entrega al cliente en un servicio óptimo.