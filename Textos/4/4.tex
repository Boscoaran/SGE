\begin{itemize}
    \item \textbf{Producto distribuible:} Actualmente, la empresa cuenta con varios productos distribuibles, entre ellos, un portátil (Portátil ASUS X415EA - EB526) y los componentes que también usamos para el montaje de los ordenadores personalizados (procesadores, gráficas, discos duros…).
    \item \textbf{Producto fabricado que se planifica a través de un PMP:} La fabricación del ordenador personalizado Typhoon X1 se realiza mediante un programa maestro de la producción.
    \item \textbf{Producto fabricado just in time:} Cuando se pide un Backplate gráfica, se automatiza el proceso, creando la consecuente orden de producción y, en caso de ser necesario, una orden de compra para los materiales. Al tratarse de un producto JIT no se tiene inventario del mismo.
    \item \textbf{Producto tipo servicio que genera un proyecto:} Hemos creado el servicio \textit{Reparación de PCs} que, una vez un cliente nos lo pida, se generará el proyecto \textit{Reparar ordenador}.
    \item \textbf{Dos productos con variantes de dos atributos:} Contamos con dos ordenadores personalizados (Typhoon X1 y Cycloon X1), los cuales tienen cuatro variantes que juntan un vinilo y una pintura. Por ejemplo, existe el PC Typhoon X1 con vinilo moderno y pintura azul, pero también existe el mismo ordenador con vinilo de videojuego y pintura azul.
    \item \textbf{Lista de materiales de, al menos, 3 materias primas en los productos fabricables:} Los ordenadores personalizados (Typhoon X1 y Cycloon X1) cuentan con ocho listas de materiales diferentes entre los dos debido a las variantes creadas. De todas formas, en todas las listas se mostrarán los 7 componentes diferentes para cada modelo pero iguales entre variantes del producto.
    \item \textbf{Jerarquías de productos:} Se han creado categorías para todos los productos de la empresa. Todos los componentes utilizados estarán en la categoría de componentes, dentro de ella, hay diferentes categorías para cada tipo de producto (disco duro, gráfica, torre…) y, finalmente, la categoría procesadores, por ejemplo, estará dividida en Intel y AMD. Las categorías también se usan para los productos fabricados y los elementos de decoración.
    \item \textbf{Dos tarifas en los productos que se compran:} Cada producto que se compra cuenta con dos tarifas con diferentes tarifas y cantidad de producto que se debe comprar. Generalmente, tendremos a Ordenadores Km0, que nos ofrecerá mejores tiempos de envío pero peores precios, y a la empresa fabricante (Intel, AMD, Asus…) que nos ofrecerán mejores precios pero, al tener almacenes fuera del estado, peores tiempos de envío.
    \item \textbf{Tarifas de venta:} Se han creado dos tarifas diferentes para nuestros productos en venta. Por un lado, tenemos la \textit{Tarifa pública}, que se aplicará a particulares y, por otro, tenemos la \textit{Tarifa empresas}, que se aplicará a empresas y ofrece un descuento con respecto de la pública puesto que las compañías realizan pedidos más grandes.
    \item \textbf{Equipos de venta:} La empresa cuenta con dos equipos de venta que desempeñarán diferentes funciones dentro del departamento. El equipo Ventas de \textit{PCs personalizados} se encargará de la gestión de los pedidos relacionados con los productos fabricados. Tendrá como miembros a Diego Marta (gerente), María Salazar, Martín Marta y José Antonio Pérez. Por otro lado, el equipo de \textit{Ventas de portátiles} se encargará de gestionar los pedidos de portátiles y otros componentes. Estará compuesto por Vicente Ayarza (gerente), Juan Segura, David Carral y Felipe Amaya.
    \item \textbf{Oportunidades de venta, pedidos en fase de presupuesto y ventas confirmadas:}
    \begin{itemize}
        \item \textbf{Oportunidades de venta:} Las oportunidades de venta creadas se encuentran en diferentes fases:
        \begin{itemize}
            \item \textbf{Nuevas:} Oportunidad Sport Medicine Castro (encargada a Diego Marta y con una tarea de envío de email) y Oportunidad de AMD (encargada a Vicente Ayarza y con una tarea de llamada).
            \item \textbf{Calificadas:} Oportunidad de MediaMarkt España (encargada a Diego Marta y con una tarea de llamada).
            \item \textbf{Propuesta:} Oportunidad de Jose Antonio Musk (encargada a David Carral).
            \item \textbf{Ganado:} Oportunidad de Juan José Zamora (asignada a Vicente Ayarza) y Oportunidad de Frutería Jesús (asignada a Juan José Zamaro).
        \end{itemize}
        \item \textbf{Pedidos en fase de presupuesto:} 
        \begin{itemize}
            \item S00002 (asignado a José Antonio Pérez con tarea de propuesta).
            \item S00003 (asignado a Juan Segura con tarea mandar presupuesto).
            \item S00010 (asignado a Vicente Ayarza).
            \item S00005 (asignado a Diego Marta con tarea de llamada cliente).
            \item S00009 (asignado a José Antonio Pérez con tarea mandar presupuesto).
            \item S00013 (asignado a Vicente Ayarza con tarea propuesta presupuesto).

        \end{itemize}
        \item \textbf{Ventas confirmadas:} existen más de 5 pedidos entre los que destacaremos S00001, S00004, S00006, S00008 y S00011.
    \end{itemize}
    \item \textbf{Descuentos en la línea de pedido:} Se han descuentos en los pedidos S00014 (10\%) y S00009 (15\%).
    \item \textbf{Costes de envío en función de la zona geográfica:} Se han creado tres tarifas de envío: el envío estándar, que se aplica a Andorra, España, Francia y Portugal; el envío a islas, que se aplica a las Islas Baleares y a las Islas Canarias; y el envío resto Europa, que se aplica a Alemania, Austria, Eslovaquia, Grecia, Holanda, Italia y la República Checa.
    \item \textbf{Órdenes de producción:} existen 5 órdenes de producción de PC Typhoon X1 y de PC Cycloon X1 (WH/MO/ 00003 (hecha), WH/MO/ 00004 (hecha), WH/MO/00005 (en progreso), WH/MO/00006 (confirmado), WH/MO/00007 (confirmado)).
    \item \textbf{Órdenes de compra:} Se han creado varias órdenes de compra entre las que destacan las siguientes: P00001 (Ordenadores Km0 encargado a Bosco Aranguren), P00002 (ASUS encargado a Lucas Gomez), P00003 (Corsair encargado a Bosco Aranguren), P00004 (Leroy Merlin España encargado a Jesús Rasines) y P00005 (Intel encargado a Bosco Aranguren).
    \item \textbf{Tareas del proyecto Reparar Ordenador:} Recepción del ordenador (departamento de almacén), búsqueda de problemas (departamento de montaje), búsqueda de herramientas (departamento de montaje), búsqueda de componentes necesarios (departamento de montaje), reparación (departamento de montaje), embalaje (departamento de almacén).
    \item \textbf{Etapas de ejecución del proyecto Reparar Ordenador:} las tareas que contienen este proyecto son Nuevas órdenes, Juntar materiales, Arreglar ordenador, A enviar.
    \item \textbf{Ventas de producto tipo servicio:} S00004 y S00005, con diferentes tareas a lo largo de las tareas del proyecto.
    \item \textbf{Horas de trabajo:} se han agregado las horas de trabajo de los empleados que han realizado sus tareas diariamente, como se puede ver en el caso de Joel Bra.
    \item \textbf{Permisos:} se han ajustado los permisos a los trabajadores dependiendo del rol que tengan en el departamento y en la empresa.
    \item \textbf{Informes:}
    \end{itemize}

    Compras por proveedor:  

    \begin{tikzpicture}
            \pie{4/ASUS,
            10/Corsair,
            6/Intel,
            58/Leroy Merlin,
            4/MSI Gaming,
            18/Ordenar KM0}
    \end{tikzpicture}

    Compras por país de origen: 
    
    \begin{tikzpicture}
            \pie{20/Estados Unidos,
            70/España,
            7/Italia,
            3/Taiwan}
    \end{tikzpicture}
    
    

