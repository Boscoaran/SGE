Usando las cinco fuerzas de Porter \textit{(Figura \ref{Porter})} podemos conocer el nivel de competencia que presenta un sector y determinar la estrategia que mejor se adapte a la situación. 

\begin{enumerate}
    \item \textbf{Amenaza de entrada de nuevos competidores potenciales:}
    
    Este apartado analiza la situación del mercado y las barrearas de entrada para nuevos competidores. En el montaje de ordenadores el producto principal por el que se paga son los componentes del ordenador más que el montaje de ellos. Debido a esto, este tipo de empresas tiene una gran parte que puede considerarse de distribución. Esto conlleva a que para poder ofrecer el mejor producto al mejor precio es necesario recibir las mejores ofertas de los proveedores. De esta característica surge la barrera de obtener componentes y dispositivos a buen precio y de gran variedad.

    \item \textbf{Amenaza de la aparición de productos sustitutivos:}
    
    En este apartado se analiza el riesgo de que un nuevo producto se haga con el mercado y sustituya el producto de nuestra empresa. Por el momento no parece posible la aparición de un producto sustitutivos del ordenador de sobremesa. Si bien los portátiles y tablets pueden llevarse una porción del mercado, ambos carecen de la capacidad y posibilidades que ofrece un ordenador de sobremesa y además ambos están disponibles como productos de nuestra empresa.

    \item \textbf{Poder de negociación de los proveedores:}
    
    Este apartado estudia el poder que tienen los proveedores a la hora de negociar los precios de los productos. En España existen muchos proveedores de componentes para ordenadores y artículos del sector de la informática. Esto es una ventaja para nosotros ya que evita que los proveedores pidan precios excesivamente altos por sus productos al tener que repartirse el mercado y ganar clientes. Como se ha mencionado en el primer apartado nos interesa tener una buena relación con nuestros proveedores y evitar que lideren las negociaciones de nuestras compras para lograr las mejores ofertas.

    \item \textbf{Poder de negociación de los clientes:}
    
    Similar al anterior apartado, este analiza el poder de negociación de los clientes sobre nuestro sector. Al igual que sucede con los proveedores, los clientes tienen un gran poder de negociación ya que España cuenta con muchas empresas que ofrecen nuestros productos y servicios. Para compensar esto debemos tener un gran representación web y ofrecer un sistema de compra online muy completo.

    \item \textbf{Rivalidad entre los competidores del sector:}
    
    El último apartado analiza las empresas rivales del sector. En España el sector esta principalmente compuesto por: PCComponentes, APP Informática, Aussar, Wipoid, Alternate y PCBox. Dependiendo de lo que el cliente busque cada una esta orientada a algo en especial (gaming, trabajo, componentes poco comunes, gamas...). Principalmente podemos dividir el mercado en dos categorías: online y física. Nuestra empresa se centrará en el mercado online ya que consideramos que nuestro producto principal no necesita de la presencia física del cliente sino de un buen servicio online.

\end{enumerate}