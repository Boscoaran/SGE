\begin{itemize}
    \item 1 producto distribuible (se compra para después vender).
    \item 1 producto fabricado cuya producción se planifica a través de un plan maestro de producción.
    \item 1 producto fabricado just in time cuya producción esté automatizada a partir del pedido de venta.
    \item 1 producto tipo servicio cuya venta generará la creación de un nuevo proyecto.
    \item Al menos dos de los productos deben incluir variantes con 2 atributos.
    \item Cada producto fabricado tiene una lista de materiales de, al menos, 3 materias primas.
    \item Se debe establecer una jerarquía de, al menos, 3 categorías de productos (aplicables también a las materias
    primas).
    \item Todos los productos que se compran deben tener, al menos, dos proveedores con sus tarifas de precios
    definidos.
    \item Todos los productos que se venden deben tener, al menos, dos tarifas de precios configuradas.
    \item Se deben definir, al menos, dos equipos de ventas con un responsable y tres comerciales a su cargo.
    \item Deben existir, al menos, 5 oportunidades de venta en curso (en distintos estados), 5 pedidos de venta en fase de presupuesto y 5 pedidos de venta confirmados. Estos pedidos deberán estar distribuidos entre los distintos equipos de ventas y empleados. El seguimiento de estos pedidos incluirá la definición de tareas como envío de emails, envío de documentación y planificación de reuniones.
    \item Al menos dos de los presupuestos ofertados deberán incluir descuentos en la línea de pedido independientemente de la tarifa aplicada.
    \item Se definirán costes de envío en función de la zona geográfica.
    \item Se deben generar distintas órdenes de fabricación como consecuencia de las ventas realizadas de los distintos productos: provenientes de ventas de productos just-in-time o de la ejecución del plan maestro de producción.
    \item De forma similar, deben aparecer las órdenes de compra derivadas de los distintos tipos de abastecimiento contemplados.
    \item El proyecto asociado al producto de tipo servicio debe tener un mínimo de 5 tareas que se asignan a responsables de, al menos, dos departamentos distintos.
    \item Se habrán definido, al menos, 3 etapas en la ejecución de los proyectos.
    \item Debe haber, al menos, dos ventas del producto de tipo servicio y los proyectos asociados deberán crearse automáticamente. Habrá tareas distribuidas a lo largo de las etapas, en función de su grado de desarrollo. El seguimiento de estos proyectos incluirá la definición de tareas como envío de emails, envío de documentación y planificación de reuniones.
    \item Las tareas se irán ejecutando incorporando las horas de trabajo de los empleados a cada tarea.
    \item Los permisos de acceso de los empleados deberán ajustarse a sus funciones (ver sólo los módulos que les competen) y su nivel de decisión (el responsable puede ver la información de los empleados a su cargo, pero no los de otros empleados).
    \item Todos los módulos implementados deberán generar informes con información sobre los distintos productos/equipos de ventas/proveedores /clientes.
\end{itemize}

Ejemplos de configuraciones definidas como EXTRAS en la rúbrica:
\begin{itemize}
    \item Costes de envío en función del peso/tamaño de los productos
    \item Definición de iniciativas de venta en el CRM
    \item Definición de permisos avanzados de usuarios distintos a los vistos en clase
    \item Producto de tipo servicio adicional que cree una tarea en un proyecto existente
    \item Creación de la tienda online como implementación de módulo extra
\end{itemize}