Explicación de cada proceso \textit{(Figura \ref{workflow})}:
\begin{itemize}
    \item \textbf{El cliente elige los componentes:} Es la acción que dará comienzo a la producción del ordenador que se va a montar. Aquí, el cliente elegirá los componentes que quiere dentro de su ordenador recibiendo sugerencias según su configuración. Una vez finalizado el proceso, se enviará al personal del departamento de montaje para la revisión.
    \item \textbf{El personal revisa los componentes elegidos:} El personal recibe la lista de componentes y periféricos dada por el cliente. Se asegura la correcta compatibilidad de todos los elementos. Si es correcta se pasará a la facturación del proyecto y, en caso negativo, se enviará al cliente un correo indicando las incompatibilidades.
    \item \textbf{Facturación al cliente:} Una vez esté todo correcto se procede a realizar la facturación; para ello se obtendrán los datos del cliente a facturar. Se determinan los productos a facturar, se establecen los posibles descuentos que haya y el medio de pago. Por último se indicará el plazo de pago.
    \item \textbf{Cobro del producto:} Se tramitará el cobro al cliente según el método de pago elegido, en caso de que no sea un cobro inmediato, se notificará al cliente sobre el plazo en el momento indicado.
    \item \textbf{Se compran las materias primas:} En primer lugar la detección de necesidad de compra, seleccionamos un producto y nos reunimos con el equipo de compras. Se especifica un presupuesto para la compra y se buscan proveedores.
    \item \textbf{Se crea la orden de montaje:} A la hora de crear una orden de montaje es necesario saber; el almacén desde el que saldrá la materia prima, el almacén destino, la cantidad de productos que se crearán y la selección del producto a crear.
    \item \textbf{Se comprueba el stock del almacén:} Se organiza, planifica y controla el conjunto de mercancías que hay en el almacén.
    \item \textbf{Se obtiene la lista de materiales:} Detallar los materiales precisos que se necesitan para la fabricación de un producto.
    \item \textbf{Recepción de la materia prima:} Se comprueba en qué estado llegan los productos así como la higiene del transporte y las instalaciones del proveedor.
    \item \textbf{Ejecutar orden de montaje:} Se ejecuta la orden creada anteriormente.
    \item \textbf{Actualizar stock de componentes:} Toma de datos de los activos, catalogación e inventario físico de activos con etiquetado y análisis de los registros.
    \item \textbf{Envío del producto:} La entrega debe realizarse de forma determinada y conforme a las condiciones establecidas en la compra.
\end{itemize}


