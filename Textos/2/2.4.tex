Para ejercer su labor, la empresa necesita un suministro continuado de componentes y periféricos para ordenador. Para obtenerlos se acudirá a las propias marcas o a otros mayoristas ya sea a nivel nacional o a nivel internacional, ya que la diversificación de proveedores es necesaria para reducir su nivel de negociación.

Por un lado destacaremos entre las propias marcas a MSI y ASUS, que serán las encargadas de abastecernos de los siguientes componentes para ordenador: tarjetas gráficas, placas base, cajas de ordenador y periféricos. Por otro lado, Intel y AMD nos venderán los procesadores, además del software para diagnóstico para empresas. En cuanto a los discos duros y otras memorias, tanto mecánicos como en esta sólido serán vendidos a nuestra empresa por Seagate. Finalmente, y terminando con las marcas, Corsair será la encargada de abastecernos de memoria ram, fuentes de alimentación y otros periféricos. A nivel nacional contamos también con Ordenadores Km0 que nos proporcionará todos los componentes para ordenador que necesitamos. Al ser un proveedor mayorista que tiene un almacén en territorio nacional, los tiempos de entrega podrían ser menores en algunos casos.

Finalmente, contamos con dos proveedores para los vinilos y la pintura que usaremos en los ordenadores personalizados. Por un lado, Leroy Merlin, y por otro, Ordenadores Km0. Ambos tienen almacenes con estos productos en territorio nacional, con lo que tendremos un buen tiempo de entrega.