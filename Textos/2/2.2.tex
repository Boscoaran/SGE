Existen diferentes tipos de estructuras organizacionales, entre las que elegiremos la estructura funcional. Esta forma cuenta con diferentes ventajas y desventajas que enumeraremos a continuación:

\begin{itemize}
    \item Cada empleado se encarga exclusivamente de sus tareas, aumentando la eficiencia y la productividad de cada trabajador.
    La supervisión de un área es más sencilla.
    \item La comunicación es mucho más directa al no haber intermediarios de por medio, ayudando a la eficiencia del equipo.
    \item Se reduce la presión sobre el individuo, ya que es más sencillo delegar y compartir responsabilidades.
    \item Pueden generar conflictos de autoridad y aumentar la rivalidad entre compañeros, ya que pueden intentar establecer su enfoque.
    \item Puede existir multiplicidad de objetivos, ocasionándose conflictos en las funciones generales de la organización.    
\end{itemize}

Nuestra empresa esta dirigida por una junta directiva formada por los gerentes de cada departamento. La toma de decisiones significativas se realiza por votación, emitiendo cada departamento un voto y contando doble el voto de los departamentos más afectados hasta un máximo de tres departamentos por votación.

Los departamentos que forman nuestra empresa se analizan a continuación:

\textbf{Departamento de Marketing:} Es uno de los pilares más importantes de la empresa debido a la necesidad de tener una buena visibilidad en Internet para alcanzar al máximo número posible de clientes.

\textbf{Departamento de Recursos Humanos:} Este departamento gestiona todo lo relacionado con las personas que trabajan en la empresa. Esta dividido en dos áreas, el de contratación y el formación del personal. Nuestro departamento de RR.HH. estará dirigido por el Responsable de Recursos Humanos cuya función es impulsar políticas de recursos humanos alineadas con los valores y estrategia de la empresa. A sus ordenes equipos de selección, formación y gestión del personal.

\textbf{Departamento de Atención al Cliente:} Establece la conexión entre la empresa y todos nuestros clientes post o pre compra, estableciendo una comunicación fluida mediante vía telefónica, por correo electrónico, mensajería o soporte técnico. Contaremos con un Director Comercial que será el encargado de asegurar un funcionamiento eficaz del departamento.

\textbf{Departamento de Logística y Almacén:} Este departamento se encarga de almacenar y gestionar los envíos de los clientes y proveedores. En algunos productos se está tratando de implementar una metodología Just In Time, para ello el papel de este departamento es crucial y puede suponer una optimización y ahorro considerable.

\textbf{Departamento de Venta:} Es el departamento encargado de gestionar los presupuestos y pedidos recibidos. Debe de estar en constante comunicación con el departamento de Almacén y Montaje. Es el único departamento que cuenta con dos gerentes.

\textbf{Departamento de Compras:} Este departamento se encarga de analizar, evaluar y negociar la compra de los artículos necesarios para la empresa. Para ello debe de estar al tanto de las existencias y necesidades y hacer un estudio de los proveedores para obtener las mejores ofertas. 

\textbf{Departamento de Mantenimiento y Mejora Continua:} Su deber es optimizar y asegurar el funcionamiento de los procesos de la empresa. Para ello analiza los procesos que se llevan a cabo y busca mejorar la eficacia y eficiencia. De este análisis extrae posibles mejoras que se implantan, revisan y mejoran constantemente.

\textbf{Departamento de Montaje:} Este departamento puede considerase el de producción ya que crea un nuevo producto a partir de materias. Debe de recibir instrucciones del departo de Ventas para organizar la producción y coordinarse con el Almacén para evitar problemas de stock.

\textbf{Departamento de Administración:} El deber de este departamento es gestionar el capital de la empresa. Sus funciones incluyen la gestión y liquidación de impuestos, la recepción y emisión de facturas, la asignación de recursos a los diferentes departamentos y la gestión general de pagos y cobros. 
