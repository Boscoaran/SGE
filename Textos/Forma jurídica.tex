España cuenta con diferentes formas jurídicas para la correcta realización de la actividad económica, entre las que se ha elegido la Sociedad de Limitada (S.L.) por la siguientes razones:

\begin{itemize}
    \item No hay límite de socios que puedan participar.
    \item La inversión que se debe realizar es razonablemente pequeña (3.000€).
    \item La forma es jurídica, no física, indicando que la cantidad de dinero perdida en caso de empeoramiento de la situación financiera. Además, no existe un dueño como tal, por tanto, la responsabilidad recae en los socios.
    \item Los impuestos debidos a Hacienda, en caso de ser autónomo, pueden escalar en cuanto aumenten los ingresos. Sin embargo, el Impuesto de Sociedades, que se aplica en nuestro caso, escala hasta un máximo del 25\%.
\end{itemize}





